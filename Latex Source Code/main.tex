\documentclass{article}
\usepackage[utf8]{inputenc}
\usepackage{graphicx}
\usepackage{xcolor}
\title{\textbf{North South University}

\Large{Department of Electrical and Computer Engineering}
\newline

PROJECT PROPOSAL BY GROUP-8
\\\LARGE\textcolor{blue}{“Zoo Management System”}
\vspace{2mm}
\\\small{DATABASE SYSTEM LAB}
\\\small{CSE311L SECTION 4 (Summer 2021)}
\\\vspace{2mm}\Large{Faculty: Intisar Tahmid Naheen}
\vspace{1mm}\Large{(Lecturer)}\vspace{1mm}
\\\Large{Lab Instructor: Nazmul Alam Dipto}
}


\author{Mohammed Ahnaf Abid 1931897642, \\*Amatullah Tyba 1911345642, \\*Musfika Tanvi Mou 1512432640}



\usepackage{natbib}
\centering
\usepackage{graphicx}

\begin{document}
\begin{figure}[h]
\centering
\includegraphics[scale=0.7]{image1}
\maketitle
\end{figure}

\newpage
\section{Introduction}
{
\flushleft
A zoological garden provides an opportunity for everyone to enter an entirely unfamiliar world of curiosity and interest, and to enlighten visitors regarding the value and need for conservation of endangered wildlife. It is found that, the zoo in our country has an amazing diversity of animals often encountered in Bangladesh. In the management of zoos: the animal details, zoo cell details, places of interest and the time schedule are important aspects to run visiting tasks smoothly. Therefore, there must be an excellent exhibition technique on the information management system. Besides, to operate a zoo functionally, there must be record keeping, management of all animal species and visitor recreational facilities which are the variables that need to be considered by the Zoo Management System.

}

\vspace{8mm}
\section{Major Issues:}
{
\flushleft
The main problem and motivations behind this management system is the lack of organized information about the zoo in Bangladesh. This project is based on the wide array of information pertaining to the animal species, zoo cells and the landmarks which are present in the zoo. For example, when we visit any zoo, we do not immediately know which animal is kept at which place that we want to see. Traditionally, for learning this information, we have to take the hassle of looking for a chart present somewhere in the zoo, but with this information system, we can easily lookup all of the interesting details about the zoo by filling in the name of the animal, such as Lion, Tiger, etc. and then by running this query on the zoo website, one can easily find interesting details about the list of animal species, where they are located and other landmarks. So, the user can easily get location directions from the website.


}

\newpage
\section{Objectives:}
{\flushleft
•	There is going to be a sign-up and login panel where the user can create an account and then login using username and password.
\\*•	Animal field is going to contain information about animals.
\\*•	Zoo cell field is going to contain information about the cages.
\\*•	Landmark field is going to contain information about places to visit such as museum, cafeteria and zoological garden.
\\*•	The users can add their favorite animals from the website to their personal favorites’ list.
\\*•	The users can view the latest ticket prices after logging in.
\\*•	The users can also view the opening and closing schedule.


}

\vspace{8mm}

\section{Target Audience:}
{\flushleft

•	Visitors who want to visit the zoo in the future.
\\*•	Children along with their guardians, who want to learn more about local zoo animals.
\\*•	Researchers who to want to look-up information about the local zoo.



}
\newpage
\section{Value Proposition:}
{\flushleft

This kind of organized zoo information system is not available in Bangladesh. This system can help visitors greatly because they can look-up all kinds of information related to the zoo from the comfort of their home. They can search up their favorite animals on the online database so they have access to a wide range of information right at their fingertips. If someone is confused about the specific opening and closing times of the zoo, they can easily figure it out from the latest timings that are shown on the website.



}

\vspace{8mm}

\section{Web Application Feature and Description:}
{\flushleft

The user will first encounter a sign-up page where they must make an account by filling up their desired username and password. Then, they will be redirected to a login page where they must enter the username and password in order to access the website.

\vspace{4mm}After logging in, the user can view the following information:\vspace{4mm}

•	Ticket prices
\\*•	Animal details
\\*•	Zoo cell details
\\*•	Landmark details
\\*•	Opening and closing timing schedule.
\\*•	Favorited list of animals




}

\newpage
\section{Tools and Resources:}
{\flushleft

•	HTML
\\*•	Bootstrap
\\*•	CSS
\\*•	PHP
\\*•	MySQL




}

\vspace{20mm}

\section{Challenges:}
{\flushleft

•	Zoo Management System is costly to develop and maintain.
\\*•	Webservers have expensive hosting fees.
\\*•	The system will not be able to handle too much web traffic.
\\*•	A complicated user interface would potentially make it hard to properly use the website.
\\*•	Too much information about the zoo and its animals would be difficult to enter into the database.




}

\end{document}
